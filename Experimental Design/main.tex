
\documentclass[12pt]{article}
\usepackage[utf8]{inputenc}
\usepackage{graphicx}
\usepackage{float}
\usepackage[margin=1in]{geometry}
\usepackage{mathtools}
\usepackage{pdfpages}
\usepackage{appendix}
\usepackage{pifont}
\usepackage{fullpage}
\usepackage{setspace}
\usepackage{float}
\usepackage{titlesec}
\usepackage{gensymb}
\usepackage{amsmath}
\usepackage{indentfirst}
\usepackage{verbatim}
\usepackage{textcomp}
\usepackage{gensymb}
\usepackage[utf8]{inputenc}
\usepackage{amsmath,amssymb}
\usepackage{graphicx}
\usepackage{subfig}
\usepackage{lipsum}% http://ctan.org/pkg/lipsum
\usepackage{xcolor}% http://ctan.org/pkg/xcolor
\usepackage{xparse}% http://ctan.org/pkg/xparse
\usepackage{relsize}
\usepackage[english]{babel}
\usepackage[utf8]{inputenc}
\usepackage{biblatex} %Imports biblatex package
\usepackage{makecell}
\addbibresource{samplebib.bib} %Import the bibliography file
\graphicspath{ {./images/} }
\usepackage{titlesec}
\graphicspath{ {./images/} }

\begin{document}
\begin{titlepage}

\newcommand{\HRule}{\rule{\linewidth}{0.2mm}}

\center
 
\textsc{\LARGE Georgia Institute of Technology}\\[0.5cm]
\textsc{\Large Microgrant Measurement Protocol}\\[0.5cm] % Major heading such as course name

\HRule \\[0.4cm]
\vspace{4cm}



\begin{minipage}{0.6\textwidth}

\vspace{5cm}

\begin{center} \large




\end{center}



\end{minipage}

\vspace{8cm}
\begin{center}
%{\large \today}\\[2cm] 
{\large February 10th, 2024.}\\[2cm] 
\end{center}



\end{titlepage}

\newpage

\section*{Executive Summary}
This document provides a measurement protocol as part of the validation process for the seamless 3D heat transfer analysis with the OpenFOAM tool. Below are the goals and the detailed experiment steps.


\section*{Goals} 
To assess the performance of the offices in Hinman and provide a better solution for insulation in the location. Also, to calculate net savings of the energy consumption after providing the new solution.

\section*{Experimental Design}
\subsection*{Steps}
\begin{enumerate}

    \item Purchase GreenTeg U-val kit
    \item Prepare setup, GreenTeg software, and equipment to attach the sensors
     \item  Select a section of the brick wall located in Hinman 359D for testing that represents typical conditions of the building envelope.
     \item  Position the sensor one on the exterior surface of the brick wall, and the second sensor on the interior surface of the wall.
    \item  Allow a minimum of 72 hrs for the system to stabilize and for the material to reach steady-state conditions.
    \item collect resulting data of calculated U-value by using GreenTeg software which also allows real-time readings
     \item  After the testing period, analyze the recorded data.
 \item Provide recommendations for improving insulation efficiency based on the measured U-value.


    
\end{enumerate}





\subsection*{Notes}
\begin{itemize}
    \item Avoid placing sensors near air vents or sources of airflow to prevent inaccurate measurements.
    \item Position sensors away from rain and direct sunlight to minimize the influence of solar radiation on readings.
\end{itemize}







\end{document}
